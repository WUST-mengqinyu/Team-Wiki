\begin{enumerate}
\item 约数定理:若$n=\prod_{i=1}^kp_i^{a_i}$,则

\begin{enumerate}
\item 约数个数$f(n)=\prod_{i=1}^k(a_i+1)$
\item 约数和$g(n)=\prod_{i=1}^k(\sum_{j=0}^{a_i}p_i^j)$
\end{enumerate}

\item 小于$n$且互素的数之和为$n\varphi(n)/2$

\item 若$\gcd(n,i)=1$,则$\gcd(n,n-i)=1(1\leq i\leq n)$

\item 错排公式:$D(n)=(n-1)(D(n-2)+D(n-1))=\sum_{i=2}^n\frac{(-1)^kn!}{k!}=[\frac{n!}{e}+0.5]$

\item 威尔逊定理:$p\ is\ prime\ \Rightarrow (p-1)!\equiv-1\pmod p$

\item 欧拉定理:$\gcd(a,n)=1\Rightarrow a^{\varphi(n)}\equiv1\pmod n$

\item 欧拉定理推广:$\gcd(n,p)=1\Rightarrow a^n\equiv a^{n\%\varphi(p)}\pmod p$

\item 模的幂公式:$a ^ n \pmod {m} = 
\begin{cases}
a ^ n \mod m & n < \varphi(m)\\
a ^ {n \% \varphi(m) + \varphi(m)} \mod m & n \ge \varphi(m)
\end{cases}
$

\item 素数定理:对于不大于$n$的素数个数$\pi(n)$,$\lim\limits_{n\to\infty}\pi(n)=\frac{n}{\ln n}$

\item 位数公式:正整数$x$的位数$N=\log_{10}(n)+1$

\item 斯特灵公式$n!\approx\sqrt{2\pi n}(\frac{n}{e})^n$

\item 设$a>1,m,n>0$,则$\gcd(a^m-1,a^n-1)=a^{\gcd(m,n)}-1$

\item 设$a>b,\gcd(a,b)=1$,则$\gcd(a^m-b^m,a^n-b^n)=a^{\gcd(m,n)}-b^{\gcd(m,n)}$

$$
G=\gcd(C_n^1,C_n^2,...,C_n^{n-1})=
\begin{cases}
	n, & \text{$n$ is prime} \\
	1, & \text{$n$ has multy prime factors} \\
	p, & \text{$n$ has single prime factor $p$}
\end{cases}
$$

$\gcd(Fib(m),Fib(n))=Fib(\gcd(m,n))$

\item 若$\gcd(m,n)=1$,则:

\begin{enumerate}
\item 最大不能组合的数为$m*n-m-n$
\item 不能组合数个数$N=\frac{(m-1)(n-1)}{2}$
\end{enumerate}

\item $(n+1)lcm(C_n^0,C_n^1,...,C_n^{n-1},C_n^{n})=lcm(1,2,...,n+1)$

\item 若$p$为素数,则$(x+y+...+w)^p\equiv x^p+y^p+...+w^p\pmod p$

\item 卡特兰数:1, 1, 2, 5, 14, 42, 132, 429, 1430, 4862, 16796, 58786, 208012

$h(0)=h(1)=1,h(n)=\frac{(4n-2)h(n-1)}{n+1}=\frac{C_{2n}^n}{n+1}=C_{2n}^n-C_{2n}^{n-1}$

\item 伯努利数:$B_n = -\frac{1}{n+1} \sum_{i=0}^{n-1} C_{n+1}^i B_i$

$$\sum_{i=1}^n i^k = \frac{1}{k+1} \sum_{i=1}^{k+1}C_{k+1}^i B_{k+1-i}(n+1)^i$$

\item 二项式反演:$$f_n = \sum_{i = 0} ^ n (-1) ^ i \binom{n}{i} g_i \Leftrightarrow g_n = \sum_{i = 0} ^ n (-1) ^ i \binom{n}{i} f_i$$
$$f_n = \sum_{i = 0} ^ n \binom{n}{i} g_i \Leftrightarrow g_n = \sum_{i = 0} ^ n (-1) ^ {n - i} \binom{n}{i} f_i$$

\item 2的n次方,在pow时可以精确输出最大2^1023,printf("%0.f\n", pow(2,1023))

\item FFT常用素数

\begin{tabular}{cccc}
    \hline
    $r⋅2^k+1$&$r$&$k$&$g$\\
    \hline
    3&1&1&2\\
    5&1&2&2\\
    17&1&4&3\\
    97&3&5&5\\
    193&3&6&5\\
    257&1&8&3\\
    7681&15&9&17\\
    12289&3&12&11\\
    40961&5&13&3\\
    65537&1&16&3\\
    786433&3&18&10\\
    5767169&11&19&3\\
    7340033&7&20&3\\
    23068673&11&21&3\\
    104857601&25&22&3\\
    167772161&5&25&3\\
    469762049&7&26&3\\
    998244353&119&23&3\\
    1004535809&479&21&3\\
    2013265921&15&27&31\\
    2281701377&17&27&3\\
    3221225473&3&30&5\\
    75161927681&35&31&3\\
    77309411329&9&33&7\\
    206158430209&3&36&22\\
    2061584302081&15&37&7\\
    2748779069441&5&39&3\\
    6597069766657&3&41&5\\
    39582418599937&9&42&5\\
    79164837199873&9&43&5\\
    263882790666241&15&44&7\\
    1231453023109121&35&45&3\\
    1337006139375617&19&46&3\\
    3799912185593857&27&47&5\\
    4222124650659841&15&48&19\\
    7881299347898369&7&50&6\\
    31525197391593473&7&52&3\\
    180143985094819841&5&55&6\\
    1945555039024054273&27&56&5\\
    4179340454199820289&29&57&3\\
    \hline
\end{tabular}

\end{enumerate}
