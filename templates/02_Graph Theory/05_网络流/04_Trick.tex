\subsubsection*{建模技巧}


\indent

\textbf{二分图带权最大独立集}。给出一个二分图,每个结点上有一个正权值。要求选出一些点,使得这些点之间没有边相连,且权值和最大。

\indent

\textbf{解:}在二分图的基础上添加源点$S$和汇点$T$,然后从$S$向所有$X$集合中的点连一条边,所有$Y$集合中的点向$T$连一条边,容量均为该点的权值。$X$结点与$Y$结点之间的边的容量均为无穷大。这样,对于图中的任意一个割,将割中的边对应的结点删掉就是一个符合要求的解,权和为所有权减去割的容量。因此,只需要求出最小割,就能求出最大权和。

\indent

\textbf{公平分配问题}。把$m$个任务分配给$n$个处理器。其中每个任务有两个候选处理器,可以任选一个分配。要求所有处理器中,任务数最多的那个处理器所分配的任务数尽量少。不同任务的候选处理器集$\lbrace p_1 , p_2 \rbrace$保证不同。

\indent

\textbf{解:}本题有一个比较明显的二分图模型,即$X$结点是任务,$Y$结点是处理器。二分答案$x$,然后构图,首先从源点$S$出发向所有的任务结点引一条边,容量等于$1$,然后从每个任务结点出发引两条边,分别到达它所能分配到的两个处理器结点,容量为$1$,最后从每个处理器结点出发引一条边到汇点$T$,容量为$x$,表示选择该处理器的任务不能超过$x$。这样网络中的每个单位流量都是从$S$流到一个任务结点,再到处理器结点,最后到汇点$T$。只有当网络中的总流量等于$m$时才意味着所有任务都选择了一个处理器。这样,我们通过$O(\log m)$次最大流便算出了答案。

\indent

\textbf{区间$k$覆盖问题}。数轴上有一些带权值的左闭右开区间。选出权和尽量大的一些区间,使得任意一个数最多被k个区间覆盖。

\indent

\textbf{解:}本题可以用最小费用流解决,构图方法是把每个数作为一个结点,然后对于权值为$w$的区间$[u,v)$加边$u→v$,容量为$1$,费用为$-w$。再对所有相邻的点加边$i→i+1$,容量为$k$,费用为$0$。最后,求最左点到最右点的最小费用最大流即可,其中每个流量对应一组互不相交的区间。如果数值范围太大,可以先进行离散化。

\indent

\textbf{最大闭合子图}。给定带权图$G$(权值可正可负),求一个权和最大的点集,使得起点在该点集中的任意弧,终点也在该点集中。

\indent

\textbf{解:}新增附加源$s$和附加汇$t$,从$s$向所有正权点引一条边,容量为权值;从所有负权点向汇点引一条边,容量为权值的相反数。求出最小割以后,$S - \lbrace s \rbrace$就是最大闭合子图。

\indent

\textbf{最大密度子图}。给出一个无向图,找一个点集,使得这些点之间的边数除以点数的值(称为子图的密度)最大。

\indent

\textbf{解:}如果两个端点都选了,就必然要选边,这就是一种推导。如果把每个点和每条边都看成新图中的结点,可以把问题转化为最大闭合子图。

\indent

\textbf{无源汇有上下界可行流:}附加源$S$和汇$T$;对于边$(u, v, min, max)$,记$d[u]-=min, d[v]+=max$,并添加弧$(u, v, max - min)$;对于流量不平衡的点$u$,设多余流量为$W$,如果$W>0$,添加弧$S->u:W$,否则若$W<0$,添加弧$u->T:-W$,求改造后的网络$S-T$最大流即可,当且仅当所有附加弧满载时原图有可行流。

\indent

\textbf{有源汇有上下界可行流:}建$t->s$,容量为inf,然后和无源汇相同。

\indent

\textbf{有源汇有上下界最大/最小流:}与上面相同,跑完可行流$S->T$后去掉边$t->s$,最大流为加$s->t$,最小流为$G[s][t].cap-max_flow(t,s)$。

